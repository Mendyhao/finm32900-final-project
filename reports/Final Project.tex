\documentclass{article}

% Language setting
% Replace `english' with e.g. `spanish' to change the document language
\usepackage[english]{babel}

% Set page size and margins
% Replace `letterpaper' with `a4paper' for UK/EU standard size
\usepackage[letterpaper,top=2cm,bottom=2cm,left=3cm,right=3cm,marginparwidth=1.75cm]{geometry}

% Useful packages
\usepackage{amsmath}
\usepackage{graphicx}
\usepackage[colorlinks=true, allcolors=blue]{hyperref}


\title{Replication of Nozawa, Yoshio's “What Drives the Cross‐Section of Credit Spreads?: A Variance Decomposition Approach.”}
\author{Julia Klauss, Joy Wu, Mengdi Hao, Yu-Ting Weng}

\begin{document}
\maketitle

\begin{abstract}
Your abstract.
\end{abstract}

\section{Introduction}

In this project, we replicate data and tables from Nozawa, Yoshio's “What Drives the Cross‐Section of Credit Spreads?: A Variance Decomposition Approach." In addition, we reproduce the data and tables with updated numbers until 2023/12/31. 

\section{ Data Collection and Preprossessing}

\subsection{Data Collection}

The panel data for corporate bond prices is constructed from four primary databases: Lehman Brothers Fixed Income Database, Mergent FISD/NAIC Database, and TRACE. The priority order for overlapping data is Lehman Brothers, TRACE, and Mergent FISD/NAIC. 

\subsection{Data Prepossessing}

The merging process involves combining Lehman Brothers and TRACE data, and filling missing values with Mergent FISD/NAIC.  As we utilize the WRDS Bond Return database, it's crucial to note that this source inherently includes monthly bond returns that account for defaults. We do not rely on Moody's Default Risk Service for complementing prices upon default. The dataset undergoes filtering to remove bonds with floating rates and non-callable options. Matching with synthetic Treasury bonds is performed to calculate excess returns and credit spreads. 

Data cleaning includes removing bonds with prices higher than matching Treasury bond prices and handling return observations showing significant bouncebacks. The final dataset is sorted into 10 columns based on yield spreads, each representing a U.S. corporate bond portfolio. This comprehensive process ensures a robust dataset for empirical analysis. 

\subsection{Difficulties}

Difficulties arose during our replication efforts, specifically in the treasury matching section. In the initial step, we conducted cubic spline interpolation to derive the Treasury yield curve and subsequently constructed Treasury zero-coupon yield curves through the bootstraping method. This provided results covering the period from July 1, 1992, to January 1, 2024, which were stored in the file "Treasury Zero Coupon Rate.csv". However, due to NaN values in the "Monthly Treasury Yield.csv" dataset, we couldn't extend this process to the earlier period from January 1, 1973, to June 1, 1992. 

\section{ Mathematics and Functions}

\subsection{How to write Mathematics}

\LaTeX{} is great at typesetting mathematics. Let $X_1, X_2, \ldots, X_n$ be a sequence of independent and identically distributed random variables with $\text{E}[X_i] = \mu$ and $\text{Var}[X_i] = \sigma^2 < \infty$, and let
\[S_n = \frac{X_1 + X_2 + \cdots + X_n}{n}
      = \frac{1}{n}\sum_{i}^{n} X_i\]
denote their mean. Then as $n$ approaches infinity, the random variables $\sqrt{n}(S_n - \mu)$ converge in distribution to a normal $\mathcal{N}(0, \sigma^2)$.


\section{ Results}

\subsection{Summary Statistic Tables}


\subsection{Figures}

Comments can be added to your project by highlighting some text and clicking ``Add comment'' in the top right of the editor pane. To view existing comments, click on the Review menu in the toolbar above. To reply to a comment, click on the Reply button in the lower right corner of the comment. You can close the Review pane by clicking its name on the toolbar when you're done reviewing for the time being.

Track changes are available on all our \href{https://www.overleaf.com/user/subscription/plans}{premium plans}, and can be toggled on or off using the option at the top of the Review pane. Track changes allow you to keep track of every change made to the document, along with the person making the change. 


\bibliographystyle{alpha}
\bibliography{sample}

\end{document}